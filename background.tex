% A simple template for LaTeX documents

\documentclass{article}

\usepackage[margin=1in]{geometry}

\newcommand{\E}{\mathrm{E}}
\newcommand{\Var}{\mathrm{Var}}
\newcommand{\Cov}{\mathrm{Cov}}
\newcommand{\Corr}{\mathrm{Corr}}


\begin{document}


\begin{description}

\item[Covariance] - Measures how much two random variables change together. We use $N - 1$ to estimate because of Bessel's correction (think degrees of freedom). The real purpose is to build up more interpretable measures like Standard Deviation and correlation.
\[
    \Cov(X,Y) = \E[(X - \E[X])(Y - \E[Y])] \approx
    \frac{1}{N - 1} \sum_{i=1}^N (x_i - \bar{x})(y_i - \bar{y})
\]

\item[Variance] - Measures the spread of a random variable. $\Var(X) = \Cov(X,X)$

\item[Standard Deviation] - Measures how close data points are to the mean. This is where the term \emph{six sigma} comes from.
\[
    \sigma_X = \sqrt{\Var(X)} \approx 
    \sqrt{\frac{1}{N-1} \sum_{i=1}^N(x_i - \bar{x})^2}
\]

\item[Correlation] specifically, the Pearson product-moment correlation coefficient, which detects linear relationships between vectors. The value is between -1 and 1. Observe how the units cancel.
\[
    \Corr(X,Y) = \frac{\Cov(X,Y)}{\sigma_X \sigma_Y}
\]

\end{description}


\end{document}
